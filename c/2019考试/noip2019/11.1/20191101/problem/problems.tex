\documentclass{article}
\usepackage{xeCJK}
\usepackage{fontspec}
\usepackage{geometry}
\usepackage{booktabs}
\usepackage{threeparttable}
\usepackage{amsmath}
\usepackage{amsfonts}
\usepackage{amssymb}
\usepackage{setspace}
\usepackage{ulem}
\usepackage{tabularx}  
\usepackage[colorlinks,linkcolor=blue,anchorcolor=blue,citecolor=green,CJKbookmarks=True]{hyperref}
\usepackage{graphicx}
\geometry{left=4.0cm,right=4.0cm,top=2.5cm,bottom=2.5cm}
\renewcommand{\normalsize}{\fontsize{12pt}{\baselineskip}\selectfont}
\begin{document}
\title{国庆水题测试}
\date{\today}
\author{blcup、jiangtao、Wahacer}
\maketitle
\begin{center}
\renewcommand{\arraystretch}{1.5}
\begin{table}[h]
\centering
\begin{tabular}{|p{3.2cm}<{\centering}|p{3.2cm}<{\centering}|p{3.2cm}<{\centering}|p{3.2cm}<{\centering}|}  
\hline
题目名称  & Azur\ Lane & function & orz \\  
\hline
提交文件名 & az & function & orz \\
\hline
输入文件名 & az.in & function.in &  orz.in\\
\hline
输出文件名 & az.out & function.out &  orz.out\\
\hline
每个测试点时限 & 2s & 1s & 1s \\
\hline
题目类型 & 传统型 & 传统型 & 传统型 \\
\hline
空间限制 & 256MB & 256MB & 162MB \\
\hline
测试点个数 & 10 & 10 & 10 \\
\hline
每个测试点分数 & 10 & 10 & 10 \\
\hline
\end{tabular}
\end{table}
\end{center}
注:
\begin{itemize}
  \item 测评环境为 lemon+windows+cpu 主频 3.4GHz
  \item 默认栈空间限制与内存限制相同
  \item 题目难度与顺序无关
  \item 仅提交一个总文件夹
\end{itemize}
\newpage
\begin{spacing}{1.0}
\section{Azur\ Lane}
\subsection{题目背景}
这是一个表面被71\%的海水覆盖着的蓝色星球,人类在这片碧蓝色之中出生、成长、孕育和发展属于自己的文明。\\
然而,伴随着它们的,还有不断膨胀的野心和欲望。表面风平浪静的世界格局之下,历史的暗流涌动。\\
终于,未知的敌人在海洋中出现,面对敌方压倒性的战力,各个阵营却仍然各自为战,最终换来的结果亦是惨痛的:人类失去了90\%以上的海域控制权,科技和生活水平急速倒退。人类对于海洋以及未知的敌人产生了深深的阴影,并且将那些将他们拖入深海黑暗之中的怪物们称之为“塞壬”。\\
数十年之后,各大阵营为了夺回曾经的辉煌,终于摒弃前嫌,联合创建了第三方军事组织“碧蓝航线”,“碧蓝航线”的宗旨在于:集结并共享来自世界各大阵营的科技与资源,建造出能够与“塞壬”抗衡的新锐舰队,夺回海域的控制权。\\
创立初期,来自各大阵营的科学家和军官们协力协作,克服了许多难关,终于将“塞壬”驱逐出了近海范围。然而好景不长,以强硬的军事化为特点的阵营“铁血”突然宣布退出组织,并且与历史对手“皇家”发生了数次激烈战斗。\\
在战斗中人们惊讶的发现,“铁血”的舰队使用了来自“塞壬”的黑科技,轻松的击破了“皇家”的精锐舰队。这一切的背后究竟是有什么不为人知的PY交易?究竟是来自人类的欲望,还是来自“塞壬”的阴谋?\\
殊不知,一个来自东方的神秘阵营亦正在酝酿着更大的风暴…\\

\subsection{题目描述}
\frame{\includegraphics[height=250pt]{Image/a.jpg}}\\
众所周知,由某站代理的$Azur Lane$是一个弹幕\sout{养老婆}模拟器.\\
对于每个舰娘,得到的途径有两种,第一种是刷图,第二种是抽船(类似于十连抽).\\
由于现在是活动池限定的时间,所以$blcup$打算把他平时攒的魔方投入到限定的活动池中池中.\\
$blcup$进行了$n$次$m$连抽,由于$blcup$知道脸比较黑,所以他想知道自己有多少次抽出的船是完全不同的.\\
比如:$AAABB$和$BABAA$是相同的,$AABAA$和$AAABB$是不同的.\\
\subsection{输入格式}
第一行,输入一个数$n$.\\
接下来$n$行代表$blcup$每次抽出船的种类(每个不同的字母代表一种船)\\
\subsection{输出格式}
输出一行一个数字,表示答案.

\subsection{样例一}
\subsubsection{Input}
5\\
AAABB\\
BBAAA\\
AABCA\\
AAAAAAAA\\
ABDAUUS\\
\subsubsection{Output}
4

\subsection{数据范围}
对于$30\%$的数据,$n\le100$。\\
对于$50\%$的数据,$n\le10000$。\\
对于$100\%$的数据,$n\le50000$。\\
保证$blcup$每次抽船不超过100艘。 \\
\newpage

\section{function}
\subsection{题目描述}
函数$F_{x,y}$满足$\begin{cases}F_{1,1}=F_{1,2}=1\\F_{1,i}=F_{1,i-1}+2*F_{1,i-2} (i\geq 3)\\F_{i,j}=\sum_{k=j}^{j+n-1}F_{i-1,k}(i\geq 2,j\geq 1)\end{cases}$\\
对于给定的每一对$N$,$M$,求$F_{m,1}\bmod 1e9+7$\\
为了满足jtyy的题面带图的愿望,特加入了这种图片.\\
\frame{\includegraphics[height=80pt]{Image/b.jpg}}\\
\subsection{输入格式}
第一行一个正整数$T$,表示$T$组数据\\
接下来$T$行,每一行两个正整数$N$和$M$.\\
\subsection{输出格式}
共$T$行,对于每一对$N$,$M$,输出$F_{m,1}\bmod 1e9+7$ \\
\subsection{样例一}
\subsubsection{Input}
2\\
2 2\\
3 3\\
\subsubsection{Output}
2\\
33\\
\subsection{样例二}
\subsubsection{Input}
5\\
6 2\\
3 5\\
6 4\\
5 2\\
2 4\\
\subsubsection{Output}
42\\
1601\\
166698\\
21\\
18\\
\subsection{数据范围}
对于$30\%$的数据,$1\leq T,N,M\leq 20$\\
对于另外的$20\%$的数据,保证$N$,$M$均为偶数\\
对于另外的$20\%$的数据,保证$N$,$M$均为奇数\\
对于$100\%$的数据,$1\leq T\leq 10000,1\leq N,M\leq 2^{63}$\\
\newpage
\section{orz}
\subsection{题目描述}
在遥远的东方,有一个神秘的民族,自称per族。他们世代居住在水面上,奉c0per为神。\\
\sout{当然还有什么女神?}.\\
每逢重大庆典,per族都会在水面上举办盛大的祭祀活动。\\
我们可以把per族居住地水系看成一个由岔口和河道组成的网络。\\
每条河道连接着两个岔口,并且水按照一个固定的方向流。显然,水系中不会有环流.\\
由于人数众多的原因,per族的祭祀活动会在多个岔口上同时举行。\\
出于对c0per的尊重,这些祭祀地点的选择必须非常慎重。\\
准确地说,per族人认为,如果水可以从一个祭祀点流到另外一个祭祀点,那么祭祀将无效。\\
族长Peper希望在保持祭祀神圣性的基础上,选择尽可能多的祭祀的地点。\\
\sout{我也要加图片!}\\
\frame{\includegraphics[height=180pt]{Image/c.jpg}}\\
\subsection{输入格式}
第一行包含两个用空格隔开的整数$N$、$M$,分别表示岔口和河道的数目,岔口从$1$到$N$编号。\\
接下来$M$行,每行包含两个用空格隔开的整数$u$、$v$,\\
描述一条连接岔口$u$和岔口$v$的河道,水流方向为自$u$向$v$。\\
\subsection{输出格式}
第一行包含一个整数K,表示最多能选取的祭祀点的个数。\\
\subsection{样例一}
\subsubsection{Input}
4 4\\
1 2\\
3 4\\
3 2\\
4 2\\
\subsubsection{Output}
2
\subsubsection{Explanation}
在样例给出的水系中,不存在一种方法能够选择三个或者三个以上的祭祀点。包含两个祭祀点的测试点的方案有两种:\\
选择岔口1与岔口3(如样例输出第二行),选择岔口1与岔口4。\\
水流可以从任意岔口流至岔口2。如果在岔口2建立祭祀点,那么任意其他岔口都不能建立祭祀点\\
但是在最优的一种祭祀点的选取方案中我们可以建立两个祭祀点,所以岔口2不能建立祭祀点。\\
对于其他岔口至少存在一个最优方案选择该岔口为祭祀点,所以输出为1011。\\
\subsection{数据范围}

\begin{center}
\renewcommand{\arraystretch}{1.5}
\begin{table}[h]
\centering
\begin{tabular}{|p{5.5cm}<{\centering}|p{5.5cm}<{\centering}|p{3.2cm}<{\centering}|p{3.2cm}<{\centering}|}  
\hline
测试点  & $n$ & $m$   \\  
\hline
对于$30\%$的数据 & $n\leq 100$ & $m=1000$ \\
\hline
对于另外$20\%$的数据 & $n= 10$ & $m\leq 40$\\
\hline
对于另外$20\%$的数据& $n=100$ & $m\leq 150$\\
\hline
对于另外$10\%$的数据&$n=40$ &$m=200$ \\
\hline
对于另外$20\%$的数据 &$n=100$ &$m\leq 800$ \\
\hline
\end{tabular}
\end{table}
\end{center}

\end{spacing}
\end{document}