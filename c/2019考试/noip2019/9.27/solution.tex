\documentclass{article}
\usepackage{ctex}
\usepackage[top=0.7in,bottom=0.7in,left=0.5in,right=0.5in]{geometry}
\usepackage{array}
\usepackage{multirow}
\usepackage{graphicx}
\usepackage{fancyhdr}
\usepackage{lastpage}
\pagestyle{fancy}
\lhead{NOIP2018模拟赛} 
\rhead{} 
\cfoot{}
\rfoot{\thepage} 
\renewcommand{\headrulewidth}{0.4pt} 
\renewcommand{\footrulewidth}{0.4pt}

\title{NOIP2018模拟赛 题解}
\date{\today}

\begin{document}

	\maketitle

	\section{考场安排}
    
    对于30\%的数据:搜索枚举所有情况。

    对于100\%的数据:
    
    比较容易想到的是贪心解法:优先考虑大班级坐大桌子即可。

    另一种解法为网络最大流,从源点向每个班级连边,容量为班级人数,从每个教室向汇点连边,容量为教室人数,从每个班级向每个教室连边,容量为\(1\),使用任何最大流算法均可。

	\section{小X的数列}
    
    对于20\%的数据,枚举情况判断即可;

    对于50\%的数据,(我也不知道怎么写);

    对于100\%的数据:

    首先考虑两端的数,如果这位\(b[i]=k\),那么这一位的期望是\(a_i\times \frac 1n\times \frac {i-1}{n-1}\);

    即两端的期望为\(a_i\times \frac 1n\times \sum_{i=1}^{n-1} \frac i{n-1}=\frac 12 a_i\);

    再考虑中间的数,如果这位\(b[i]=k\),那么这一位的期望是\(a_i\times \frac 1n\times \frac {i-1}{n-1}\times \frac {i-2}{n-2}\);

    即中间的期望是\(a_i\times \frac 1n\times \sum_{i=1}^{n-2} \frac {i(i+1)}{(n-1)(n-2)}=\frac 13 a_i\);

    提示:\(\sum_{i=1}^n i^2=\frac {n(n+1)(2n+1)}6\)。

    即答案为\(\frac 12\times (a_1+a_n)+\frac 13\times \sum_{i=2}^{n-1} a_i\)。
    
	\section{警力覆盖}
    
	对于30\%的数据,二进制枚举;

    对于100\%的数据,考虑树形DP:

    设\(f[u][0]\)表示在\(u\)村庄建设警务站,其父亲和儿子是否建设警务站均可;

    设\(f[u][1]\)表示在\(u\)的父亲建设警务站,不在\(u\)和\(u\)的儿子建设警务站;

    设\(f[u][2]\)表示在\(u\)的一个儿子建设警务站,不在\(u\)和\(u\)的父亲建设警务站;

    设\(v\)是\(u\)的一个儿子:

    从叶节点向根转移,容易得出状态转移方程\(f[u][0]=\sum_v min(f[v][0],f[v][1]),f[u][1]=\sum_v f[v][2]\);

    考虑\(f[u][2]\)和状态\(f[u][1]\)的不同:

    \(f[u][2]\)要求在\(v\)建设警务站,\(u\)和\(v\)的儿子不建设;

    \(f[u][1]\)要求在\(v\)的一个儿子建设警务站,\(u\)和\(v\)的不建设;

    故用\(f[u][1]\)减去\(v\)的儿子建设警务站的情况(\(f[v][2]\)),加上\(v\)建设警务站的情况(\(f[v][0]\))即可。

\end{document}